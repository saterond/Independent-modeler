\chapter{Introduction}

Nowadays in the software engineering the main emphasis is (in addition to the customer needs and in addition to meet the defined development time, price and the desired quality) on the reusability, maintainability and testability of the software, its rapid development, etc. One important part of the software development is the model creation (or modeling). The model depicts the most important parts of the software and helps the developer to understand the problematic. Probably the most known and the most used model is the UML\footnote{UML - Unified Modeling Language} Class Diagram. There are a lot of tools which are able to create it but very often they can't be used for free.

This bachelor thesis deals with an implementation of Netbeans plugin which will be used for class modeling. I chose this theme because of two main reasons. The first reason is that it is not a standard information system (or something similar) that can be created with no need to learn new technologies (these systems are based on usage of well-known common frameworks). In this project I will learn to work with new frameworks like JGraph (section \ref{section:JGraph}) and Netbeans Platform (section \ref{section:netbeansAPI}). The second reason is that this project can produce an useful system which can be extended e.g. within the scope of another thesis. Of course, the result of one bachelor thesis cannot produce more sophisticated system than the existing (paid) ones, but it provides a good basis for further expansion. This project is a part of the bigger whole - there are several projects that creates other modeling tools. These tools could be joined together in the future.

At the beginning I would like to point out that this work is not a recherche one. I didn't try to compare several possible solutions of the problem but I tried to create a solution of the problem. Despite the fact that I have been inspired by several existing systems, these systems are not described in a detail. In this thesis I focused on the description of the implementation and thus it should be easy to implement similar system according to it.

The structure of this thesis is divided into two main parts. The first part is called Analysis. This part contains the information about the functionality of the program from the view of the user, a little info about existing solutions that inspired me, used frameworks and some of used design patterns. The second part is called Application and consists of implementation documentation. In this part there is a description of how the system works inside and of some interesting subsystems.
