%% History:
% Pavel Tvrdik (26.12.2004)
%  + initial version for PhD Report
%
% Daniel Sykora (27.01.2005)
%
% Michal Valenta (3.12.2008)
% rada zmen ve formatovani (diky M. Duškovi, J. Holubovi a J. Žďárkovi)
% sjednoceni zdrojoveho kodu pro anglickou, ceskou, bakalarskou a diplomovou praci

% One-page layout: (proof-)reading on display
%%%% \documentclass[11pt,oneside,a4paper]{book}
% Two-page layout: final printing
\documentclass[11pt,twoside,a4paper]{book}   
%=-=-=-=-=-=-=-=-=-=-=-=--=%
% The user of this template may find useful to have an alternative to these 
% officially suggested packages:
\usepackage[czech, english]{babel}
\usepackage[T1]{fontenc} % pouzije EC fonty 
% pripadne pisete-li cesky, pak lze zkusit take:
% \usepackage[OT1]{fontenc} 
\usepackage[utf8]{inputenc}
%=-=-=-=-=-=-=-=-=-=-=-=--=%
% In case of problems with PDF fonts, one may try to uncomment this line:
%\usepackage{lmodern}
%=-=-=-=-=-=-=-=-=-=-=-=--=%
%=-=-=-=-=-=-=-=-=-=-=-=--=%
% Depending on your particular TeX distribution and version of conversion tools 
% (dvips/dvipdf/ps2pdf), some (advanced | desperate) users may prefer to use 
% different settings.
% Please uncomment the following style and use your CSLaTeX (cslatex/pdfcslatex) 
% to process your work. Note however, this file is in UTF-8 and a conversion to 
% your native encoding may be required. Some settings below depend on babel 
% macros and should also be modified. See \selectlanguage \iflanguage.
%\usepackage{czech}  %%%%%\usepackage[T1]{czech} %%%%[IL2] [T1] [OT1]
%=-=-=-=-=-=-=-=-=-=-=-=--=%

%%%%%%%%%%%%%%%%%%%%%%%%%%%%%%%%%%%%%%%
% Styles required in your work follow %
%%%%%%%%%%%%%%%%%%%%%%%%%%%%%%%%%%%%%%%
\usepackage{graphicx}
%\usepackage{indentfirst} %1. odstavec jako v cestine.

\usepackage{k336_thesis_macros} % specialni makra pro formatovani DP a BP
 % muzete si vytvorit i sva vlastni v souboru k336_thesis_macros.sty
 % najdete  radu jednoduchych definic, ktere zde ani nejsou pouzity
 % napriklad: 
 % \newcommand{\bfig}{\begin{figure}\begin{center}}
 % \newcommand{\efig}{\end{center}\end{figure}}
 % umoznuje pouzit prikaz \bfig namisto \begin{figure}\begin{center} atd.

\newcommand\luckyparindent{1.2cm}
\parindent=\luckyparindent

\usepackage{amsmath}
\usepackage{dirtree}
\usepackage{xcolor}
\usepackage{pifont}
\usepackage{url}
\usepackage{aeguill}
\usepackage{subfigure}
\usepackage{listings}


%%%%%%%%%%%%%%%%%%%%%%%%%%%%%%%%%%%%%
% Zvolte jednu z moznosti 
% Choose one of the following options
%%%%%%%%%%%%%%%%%%%%%%%%%%%%%%%%%%%%%
%\newcommand\TypeOfWork{Diplomová práce} \typeout{Diplomova prace}
% \newcommand\TypeOfWork{Master's Thesis}   \typeout{Master's Thesis} 
% \newcommand\TypeOfWork{Bakalářská práce}  \typeout{Bakalarska prace}
 \newcommand\TypeOfWork{Bachelor's Project}  \typeout{Bachelor's Project}


%%%%%%%%%%%%%%%%%%%%%%%%%%%%%%%%%%%%%
% Zvolte jednu z moznosti 
% Choose one of the following options
%%%%%%%%%%%%%%%%%%%%%%%%%%%%%%%%%%%%%
% nabidky jsou z: http://www.fel.cvut.cz/cz/education/bk/prehled.html

%\newcommand\StudProgram{Elektrotechnika a informatika, dobíhající, Bakalářský}
%\newcommand\StudProgram{Elektrotechnika a informatika, dobíhající, Magisterský}
% \newcommand\StudProgram{Elektrotechnika a informatika, strukturovaný, Bakalářský}
% \newcommand\StudProgram{Elektrotechnika a informatika, strukturovaný, Navazující magisterský}
% \newcommand\StudProgram{Softwarové technologie a management, Bakalářský}
% English study:
% \newcommand\StudProgram{Electrical Engineering and Information Technology}  % bachelor programe
% \newcommand\StudProgram{Electrical Engineering and Information Technology}  %master program
\newcommand\StudProgram{Software Technologies and Management}


%%%%%%%%%%%%%%%%%%%%%%%%%%%%%%%%%%%%%
% Zvolte jednu z moznosti 
% Choose one of the following options
%%%%%%%%%%%%%%%%%%%%%%%%%%%%%%%%%%%%%
% nabidky jsou z: http://www.fel.cvut.cz/cz/education/bk/prehled.html

%\newcommand\StudBranch{Výpočetní technika}   % pro program EaI bak. (dobihajici i strukt.)
%\newcommand\StudBranch{Výpočetní technika}   % pro prgoram EaI mag. (dobihajici i strukt.)
%\newcommand\StudBranch{Softwarové inženýrství}            %pro STM
%\newcommand\StudBranch{Web a multimedia}                  % pro STM
\newcommand\StudBranch{Computer Engineering}              % bachelor programe
%\newcommand\StudBranch{Computer Science and Engineering}  % master programe


%%%%%%%%%%%%%%%%%%%%%%%%%%%%%%%%%%%%%%%%%%%%
% Vyplnte nazev prace, autora a vedouciho
% Set up Work Title, Author and Supervisor
%%%%%%%%%%%%%%%%%%%%%%%%%%%%%%%%%%%%%%%%%%%%

\newcommand\WorkTitle{Netbeans plugin for class modelling}
\newcommand\FirstandFamilyName{Luk\'{a}\v{s} Vyhl\'{i}dka}
\newcommand\Supervisor{Ing. Ond\v{r}ej Macek}


% Pouzijete-li pdflatex, tak je prijemne, kdyz bude mit vase prace
% funkcni odkazy i v pdf formatu
\usepackage[
pdftitle={\WorkTitle},
pdfauthor={\FirstandFamilyName},
bookmarks=true,
colorlinks=true,
breaklinks=true,
urlcolor=red,
citecolor=blue,
linkcolor=blue,
unicode=true,
]
{hyperref}



% Extension posted by Petr Dlouhy in order for better sources reference (\cite{} command) especially in Czech.
% April 2010
% See comment over \thebibliography command for details.

\usepackage[square, numbers]{natbib}             % sazba pouzite literatury
%\usepackage{url}
%\DeclareUrlCommand\url{\def\UrlLeft{<}\def\UrlRight{>}\urlstyle{tt}}  %rm/sf/tt
%\renewcommand{\emph}[1]{\textsl{#1}}    % melo by byt kurziva nebo sklonene,
\let\oldUrl\url
\renewcommand\url[1]{<\texttt{\oldUrl{#1}}>}




\begin{document}

%%%%%%%%%%%%%%%%%%%%%%%%%%%%%%%%%%%%%
% Zvolte jednu z moznosti 
% Choose one of the following options
%%%%%%%%%%%%%%%%%%%%%%%%%%%%%%%%%%%%%
%\selectlanguage{czech}
\selectlanguage{english} 

% prikaz \typeout vypise vyse uvedena nastaveni v prikazovem okne
% pro pohodlne ladeni prace


\iflanguage{czech}{
	 \typeout{************************************************}
	 \typeout{Zvoleny jazyk: cestina}
	 \typeout{Typ prace: \TypeOfWork}
	 \typeout{Studijni program: \StudProgram}
	 \typeout{Obor: \StudBranch}
	 \typeout{Jmeno: \FirstandFamilyName}
	 \typeout{Nazev prace: \WorkTitle}
	 \typeout{Vedouci prace: \Supervisor}
	 \typeout{***************************************************}
	 \newcommand\Department{Katedra počítačů}
	 \newcommand\Faculty{Fakulta elektrotechnická}
	 \newcommand\University{České vysoké učení technické v Praze}
	 \newcommand\labelSupervisor{Vedoucí práce}
	 \newcommand\labelStudProgram{Studijní program}
	 \newcommand\labelStudBranch{Obor}
}{
	 \typeout{************************************************}
	 \typeout{Language: english}
	 \typeout{Type of Work: \TypeOfWork}
	 \typeout{Study Program: \StudProgram}
	 \typeout{Study Branch: \StudBranch}
	 \typeout{Author: \FirstandFamilyName}
	 \typeout{Title: \WorkTitle}
	 \typeout{Supervisor: \Supervisor}
	 \typeout{***************************************************}
	 \newcommand\Department{Department of Computer Science and Engineering}
	 \newcommand\Faculty{Faculty of Electrical Engineering}
	 \newcommand\University{Czech Technical University in Prague}
	 \newcommand\labelSupervisor{Supervisor}
	 \newcommand\labelStudProgram{Study Programme} 
	 \newcommand\labelStudBranch{Field of Study}
}




%%%%%%%%%%%%%%%%%%%%%%%%%%    Poznamky ke kompletaci prace
% Nasledujici pasaz uzavrenou v {} ve sve praci samozrejme 
% zakomentujte nebo odstrante. 
% Ve vysledne svazane praci bude nahrazena skutecnym 
% oficialnim zadanim vasi prace.

%%%%%%%%%%%%%%%%%%%%%%%%%%    Titulni stranka / Title page 

\coverpagestarts

%%%%%%%%%%%%%%%%%%%%%%%%%%%    Podekovani / Acknowledgements 

\acknowledgements
\noindent
%Pod�kov�n�


%%%%%%%%%%%%%%%%%%%%%%%%%%%   Prohlaseni / Declaration 

\declaration{In Prague, \today}
%\declaration{In Kořenovice nad Bečvárkou on May 15, 2008}


%%%%%%%%%%%%%%%%%%%%%%%%%%%%    Abstract 
 
\abstractpage

An abstract

% Prace v cestine musi krome abstraktu v anglictine obsahovat i
% abstrakt v cestine.
\vglue60mm

\noindent{\Huge \textbf{Abstrakt}}
\vskip 2.75\baselineskip

\noindent
Abstrakt práce by měl velmi stručně vystihovat její podstatu. Tedy čím se práce zabývá a co je jejím výsledkem/přínosem.

\noindent
Očekávají se cca 1 -- 2 odstavce, maximálně půl stránky.

%%%%%%%%%%%%%%%%%%%%%%%%%%%%%%%%  Obsah / Table of Contents 

\tableofcontents


%%%%%%%%%%%%%%%%%%%%%%%%%%%%%%%  Seznam obrazku / List of Figures 

\listoffigures


%%%%%%%%%%%%%%%%%%%%%%%%%%%%%%%  Seznam tabulek / List of Tables

\listoftables


%**************************************************************

\mainbodystarts
% horizontalní mezera mezi dvema odstavci
%\parskip=5pt
%11.12.2008 parskip + tolerance
\normalfont
\parskip=0.2\baselineskip plus 0.2\baselineskip minus 0.1\baselineskip

% Odsazeni prvniho radku odstavce resi class book (neaplikuje se na prvni 
% odstavce kapitol, sekci, podsekci atd.) Viz usepackage{indentfirst}.
% Chcete-li selektivne zamezit odsazeni 1. radku nektereho odstavce,
% pouzijte prikaz \noindent.

%**************************************************************

% Pro snadnejsi praci s vetsimi texty je rozumne tyto rozdelit
% do samostatnych souboru nejlepe dle kapitol a tyto potom vkladat
% pomoci prikazu \include{jmeno_souboru.tex} nebo \include{jmeno_souboru}.
% Napr.:
% \include{1_uvod}
% \include{2_teorie}
% atd...

%*****************************************************************************


\chapter{Introduction}

Nowadays in the software engineering the main emphasis is (in addition to the customer needs and in addition to meet the defined development time, price and the desired quality) on the reusability, maintainability and testability of the software, its rapid development, etc. One important part of the software development is the model creation (or modeling). The model depicts the most important parts of the software and helps the developer to understand the problematic. Probably the most known and the most used model is the UML\footnote{UML - Unified Modeling Language} Class Diagram. There are a lot of tools which are able to create it but very often they can't be used for free.

This bachelor thesis deals with an implementation of Netbeans plugin which will be used for class modeling. I chose this theme because of two main reasons. The first reason is that it is not a standard information system (or something similar) that can be created with no need to learn new technologies (these systems are based on usage of well-known common frameworks). In this project I will learn to work with new frameworks like JGraph (section \ref{section:JGraph}) and Netbeans Platform (section \ref{section:netbeansAPI}). The second reason is that this project can produce an useful system which can be extended e.g. within the scope of another thesis. Of course, the result of one bachelor thesis cannot produce more sophisticated system than the existing (paid) ones, but it provides a good basis for further expansion. This project is a part of the bigger whole - there are several projects that creates other modeling tools. These tools could be joined together in the future.

At the beginning I would like to point out that this work is not a recherche one. I didn't try to compare several possible solutions of the problem but I tried to create a solution of the problem. Despite the fact that I have been inspired by several existing systems, these systems are not described in a detail. In this thesis I focused on the description of the implementation and thus it should be easy to implement similar system according to it.

The structure of this thesis is divided into two main parts. The first part is called Analysis. This part contains the information about the functionality of the program from the view of the user, a little info about existing solutions that inspired me, used frameworks and some of used design patterns. The second part is called Application and consists of implementation documentation. In this part there is a description of how the system works inside and of some interesting subsystems.


\chapter{Class Diagram}
\label{section:classDiagramDescription}

A class diagram is a basic UML (\cite{UMLWeb}) diagram. It is probably the best known of all UML diagrams. The class diagram is used for description of system's (or program's) structure by showing its classes and their relations. 

Despite the fact that the diagram is called class diagram, it does not describe only the classes of the system. It describes all the data types like interfaces, enumeration types, etc. Concrete types (class, interface, etc.) are distinguished by stereotypes. All these data types are in the class diagram displayed as rectangles. These rectangles can be divided into three parts: 
\begin{itemize}
    \item The first part contains the name of the class (or interface, enumeration, etc.) and eventually the stereotype (a text in \guillemotleft angle quotes\guillemotright) which describes a special property (interface, enumeration, or something else). If there is no stereotype, it means that the rectangle represents a simple class.
    \item The second part contains the list of attributes. Every attribute defines its scope (public, private, protected), name and the data type. There don't have to be all the attributes in this part, but only the important ones.
    \item The last, third part, contains the list of methods. Every method defines its scope, name, list of parameters and the return value. Again, there can be only methods which are important to depict a concrete problem.
\end{itemize}

\begin{figure}[!ht]
\begin{center}
\includegraphics[]{img/classDiagramExample.png}
\caption{Example of a class diagram}
\label{f-classDiagramExample}
\end{center}
\end{figure}

Example of a class diagram can be seen in the Figure \ref{f-classDiagramExample}. There are two rectangles. First, with name IBarking, represents the interface (it has the \guillemotleft{}interface\guillemotright stereotype). This interface defines one public method with no return type (void) called bark. The second rectangle represents simple class (there is no stereotype) called Dog. The Dog class defines three public attributes (weight of the double type, name of the String type and age of the integer type) and two public methods (pee and bark), both with no return type. Between these two rectangles there is a relationship defining that the Dog class implements the IBarking interface.

As it has been already said in the last paragraph, there can be relations between classes. UML Class diagram defines 5 basic relations between classes (or interfaces and enumerations) which define some feature of the class. These relations are:

\begin{itemize}
    \item Association - this relation type defines a feature of the source class\footnote{In this chapter when I write the source or the target class, I will not mean only a class, but also an interface or an enumeration. If it is not, I will point it out.}. In the class diagram it is depicted as a solid line which leads from the source class to the target class (the target class defines the data type of the feature). The relation can be unidirectional (only source class has the pointer to the target) or bidirectional (both source and target class has the pointer to each other). The relation can also have a simple arrow defining the relation's direction (unidirectional or bidirectional). On the end of the direction (the side where the arrow is) there is the name and cardinality of the feature. You can see an association example in the Figure \ref{f-exampleAssociation}.
    \item Aggregation - defines that the target class is a part of the source class. This relation is depicted as a solid line with a blank diamond on the start (on the side where the starting/owning class is). An example of the Aggregation is shown in the Figure \ref{f-exampleAggregation}.
    \item Composition - represents stronger relation than the Aggregation. Composition tells us that when the owner object (instance of the class) is destroyed, the owned object will be destroyed as well. Composition is depicted by a solid line with a filled diamond on the start. Figure \ref{f-exampleComposition} shows an example of the Composition.
    \item Generalization - this relation defines that source class is derived from the target class. There are some restrictions for this relation. Class can be derived only from class (not from interface) and interface can be derived again only from interface. Realisation is depicted as a solid line with a solid arrow on the end. An example of this relation can be found in the figure \ref{f-exampleGeneralization}
    \item Realisation - defines that a class (not an interface) implements an interface. Realisation is shown as a dashed line with a solid arrow on the end. The example is shown in the Figure \ref{f-exampleRealisation}
\end{itemize}

\begin{figure}[!ht]
\begin{center}
\subfigure[An Association example]{
\includegraphics[scale=1]{img/exampleAssociation.png}
\label{f-exampleAssociation}
}
\subfigure[An Aggregation example]{
\includegraphics[scale=1]{img/exampleAggregation.png}
\label{f-exampleAggregation}
}
\subfigure[A Composition example]{
\includegraphics[scale=1]{img/exampleComposition.png}
\label{f-exampleComposition}
}
\subfigure[A Generalization example]{
\includegraphics[scale=1]{img/exampleGeneralization.png}
\label{f-exampleGeneralization}
}
\subfigure[A Realisation example]{
\includegraphics[scale=1]{img/exampleRealisation.png}
\label{f-exampleRealisation}
}
\caption{Relations examples}
\label{f-relationExamples}
\end{center}
\end{figure}

\section{Meta-Model of a Class Model}
\label{classModelMetaModel}

The UML defines a notation and a meta-model. The notation is the graphical representation which can be seen in diagrams (e.g. in the class diagram). The meta-model is used to define the concepts of a model. It is represented by a diagram (very often it is a class diagram) which depicts all of its components. Meta-model is very important for developers of CASE tools because it defines the UML structure, which can be translated into the code. Person, which only creates (paints) the diagram does not need to be bothered by the existence of meta-model.

The UML's meta-model is very complex and is not shown here. It depicts the UML structure on very abstract level. Due to this fact it can be quite easily extended. For information about class model meta-model I used the information from \cite{UMLDistilled} and \cite{ArlowUML}. Some information can be found also in \cite{UMLWeb}.

\section{Platform Independend and Platform Specific Model}
\label{ClassDiagramDiagramTypes}

In a computer engineering, class diagram is used for both conceptual modeling (business modeling) and class modeling on platform level, which is consequently translated into a programming language (into the programming code).

Conceptual modeling is frequently used in early stage of software/system development. Its purpose is to depict the basic ideas of the domain that we want to build (a problem domain). It is why is this model often called Domain Object Model or simply domain model. The domain model represents the vocabulary and key features of the problem domain. These features are represented by entities that occur in the problem domain, their attributes and relations between them. This type of model is really useful as a communication tool between several members of business team or between the business and technical team. By this model it can be also verified that the system meets the user's requirements, but only if this model is created well.

Class modeling on platform level describes the physical structure of the software. It uses all elements of UML class diagram, not only classes, its attributes and relations between them. The platform is meant a specific programming language like Java or C\#. Each platform has its own features, e.g. the names of standard data types. If the class diagram is meant to be translated into the code, it has to include these platform specific features. More information about class diagram can be found in \cite{UMLDistilled}.

\section{Class Modeling Tools}

There are a lot of tools that provides class modeling. Some of them can be used for free, some of them not (most of the better). This section will show several of them.

The application that inspired me most was Enterprise Architect from Sparx Systems \cite{sparxsystemsweb} company. This application does not provide only class modeling functionality. It is a CASE\footnote{CASE - Computer Aided Software/System Engineering} system which is used for purposes of analysis and implementation. The Enterprise Architect provides all UML diagrams (Class diagram, Activity diagram, Use Case diagram, Sequence diagram, etc.) creation and much more. But only the Class diagram is important for this project because it is what this thesis deals with. The class modeling in Enterprise Architect has a lot of functionality that cannot be created during one bachelor thesis. Thus, it has to be decided which functionality will be implemented and which will not. The things that inspired me were e.g. the tool chooser panel or the way of class (or interface/enumeration) editing. Next list shows several tools which provides a class diagram modeling.

\begin{itemize}
    \item Enterprise Architect - Professional CASE tool for easy creation of UML diagrams and much more. More info in \cite{sparxsystemsweb}.
    \item ArgoUML - Open Source UML Modeling tool based on Java programming language. More info in \cite{ArgoUMLWeb}.
    \item MagicDraw - Business process, architecture, software and system modeling tool with teamwork support. More info in \cite{MagicDrawWeb}.
    \item StarUML - Open Source UML Modeling tool running on Win32 platform. More info in \cite{StarUMLWeb}.
\end{itemize}

\chapter{Analysis}

The analysis is a very important part of the whole project. It helps us to understand what is excepted from the system and how we will try to fulfil these expectations. There are many methodologies in area of software engineering but you have to do the analysis before you start to write the code no matter which methodology you choose. The analysis doesn't have to be big (especially in an agile methodology) but it has to be there.

In this chapter there is analysis of desired functionality, used frameworks and development environment. The most important part is the analysis of desired functionality because it helps to understand what is and what is not excepted from this project. The analysis of used frameworks is based on desired functionality because used framework has to be able to help with its realisation.

\section{Desired Functionality}

There are a lot of applications that provides the class modeling. Basic appearance is mostly common. There is a space where the model is shown (this space will be called workspace) and some toolbox where user can choose the tool he wants.

Every class diagram modeller has to be able to depict the basic elements of the class diagram as were described in section \ref{section:classDiagramDescription}. Existing modellers provides another functionality like information about complexities of single classes, their constraints, statuses, requirements, notes and many others. It is a lot of functionality that can not be created during one bachelor project but it can be implemented in a related project.

Main aim of this project is to implement a metamodel of the class diagram. Developers of CASE system does not treat UML diagrams like diagrams. For them, the basics of UML diagram is the metamodel and the diagram is only its graphical representation. The metamodel represents the structure of the diagram and is standardly represented by an UML class diagram. The metamodel is not used only in conjunction with UML diagrams. The metamodel can be used to represent the abstraction of any other problem. The reason why I use the metamodel in conjunction with the UML diagram is that the OMG\footnote{OMG - Object Management Group - is a consortium which created (among others) the UML specification.} defined its UML standard by use of metamodel. The structure of the metamodel that will be implemented in this project is described in next section \ref{section-metamodel}.

A feature that is not a common one and is implemented in the Indepmod Class Notation Modeller is an annotation modelling. It provides the ability to annotate a class, a method or an attribute. Annotations are used in the Java programming language from version 1.5 and in the C\# programming language. Annotations are form of metadata which are stored in the source code. This metadata specifies another features of that class/method/attribute. Alternative to annotations are configuration files (very often of XML type) that hold these information. Annotations (or XML configuration files) are often used by frameworks. An example can be the JPA\footnote{JPA - Java Persistence API} framework of Java EE\footnote{Java EE - Java Enterprise Edition} that uses these metadata for the definition of object relation mapping (ORM)\footnote{Object Relation Mapping - a technique that maps objects of object oriented software into tables of relation database}. In a former version of java, these metadata were stored in XML configuration files (very often inside one XML file). When there was a lot of classes to be configured, these configuration files were very large and confusing. On the other hand, annotations allow to specify these configurations inside the classes. This means that if you want to see or edit a class configuration, you don't have to search it in a file with maybe hundreds or thousands lines. Instead, you will open a desired class and find these information right there.

Annotations are nowadays very often used. The annotation modeling feature brings the advantage that a developer can create these annotations during a class modelling. When he creates the code (lets the case tool to generate it) from the diagram he will have these annotation right there. If the case system were not able to model annotations, the developer would have to take a look at the diagram, remember which annotations should be where (in where class/method/attribute) and add them into the code manually. Of course, this should be repeated when the model is changed.

The Indepmod Class Diagram Plugin will be also able to provide its diagram data to another Netbean's plugin. It means that there could be a plugin which will be able to e.g. generate a code according to the model or create an report. This will be done by some public API\footnote{API - Application Programming Interface} which will be provided by this application. This API will provide the type of the diagram (class or business diagram), the information about elements of the diagram and the picture of the diagram. The picture can be used by another plugin for example for a report. Concrete implementation of the API will be described in the Architecture chapter, concrete in the section \ref{section:classModelAPI}.

At the end of this section I would like to recapitulate three main requirements on the Indepmod Class Diagram Plugin. These requirements are:
\begin{itemize}
    \item Implement the standard class diagram modelling tool,
    \item it will also have a special feature for annotation modelling,
    \item created model will be provided to other Netbeans plugin by an public API.
\end{itemize}

\subsection{Class Diagram Metamodel}
\label{section-metamodel}

\begin{figure}[!ht]
\begin{center}
\includegraphics[scale=1]{img/classDiagramMetamodel.png}
\caption{Class Diagram Metamodel}
\label{f-classDiagramMetamodel}
\end{center}
\end{figure}

Figure \ref{f-classDiagramMetamodel} shows the metamodel of the class diagram that Indepmod Class Notation uses. From this metamodel you can see what the Indepmod Class Notation can and what it can not to model. Main building block of whole class diagram is an Element. The Element can be a Class, an Interface or an Enumeration. Every Element has a visibility, a stereotype (which can be empty) and a name. Each Element defines a data type identified by its name. In addition to the stereotype and the name, the Element has also the list of annotations, attributes and methods.

An Attribute has a name, a visibility and represents a data type. This data type can be the data type of an Element that we have already created or it can be another else data type (like String or int data type in Java). The Attribute can also have several (or none) Annotations. A Method is also defined by a name and has a visibility. The Attributes owned by the Method represents the parameters and the data type represents what the method returns. An Annotation has a name and can have several Annotation Attributes. Each Annotation Attribute has the name and a list of values.

\subsection{Model Type}

The Indepmod Class Notation Plugin will allow to create two types of class model. The first one is a standard Class model which has been already described in the section \ref{section:classDiagramDescription}. The second one is a business model. The selection of desired diagram in the application type will be done when user creates a new diagram. 

The purpose of the business model has been also described in the section \ref{section:classDiagramDescription}. Business model does not use all elements of the class model. It uses classes, its attributes and relations between classes. A class represents an entity from the problem domain. An attribute describes a property of the entity and a relation represents exactly what its name says - the relation between entities. An example of the business model can be the Figure \ref{f-classDiagramMetamodel}, which describes the problem domain of the class model.

\chapter{Implementation Platform and Frameworks}

\section{Netbeans API}
\label{section:netbeansAPI}

All desktop applications have several common things. They use for example windows, menus (file, edit, etc.), docking panels, help system, online update system, etc. This functionality can be created over and over again for each new desktop application but it takes a lot of time and effort to create it. And this is the reason why the frameworks are used. Advantage is that a developer (or a group of developers) creates and updates the framework that solves certain problem. Another developer can then use this framework so he does not have to implement it again. If the developer solved the problem himself, he would either create not so robust solution or spend a lot of time (maybe years) to create it. Another advantage of framework using is that it is better tested and errors are repaired in one place. If the framework is used by lot of developers, they can find another bugs, which was not found during the testing part of development, and report them to the framework's creator. He can consequently repair the bug and provide another version of the framework.

The Netbeans platform is an open source application framework based on well known Swing Technology. It can be used to simplify the desktop application development by providing many techniques and design patterns. The Netbeans platform gives the developer a tool to create a robust desktop application much faster and better. Thus, the developer can focus on the application's business logic creation instead of dealing with the work that has been already done.

Basic concept of the Netbeans API is that the application can be divided into several loosely coupled modules. This is very useful especially when the application starts to be complex. Netbeans API uses a virtual filesystem (hierarchical structure of directories and files), also called as central registry, for its configuration. Every module which is deployed into the application can add some information into this virtual filesystem and thus it can add some functionality. This is described later in the section \ref{section:toolChooserModule}.

Another useful feature of the Netbeans platform is the extended visibility settings. We can set which parts of a module will be public and which will not. This means that we can use the public visibility inside the module and specify which package will be visible outside the module. The result of this is that we can access properties of a class directly and we don't have to worry that some other package could do the same thing.

The Netbeans platform can be used either to create the whole desktop application or to create a plugin for an existing one. Both whole applications and plugins can be consisted of several (or of many) independent modules. For this project I chose the creation of a plugin. The Netbeans platform provides a lot of functionality and if you want to know more about it, please take look at \cite{netbeans6.9DevGuide}.

\section{JGraph}
\label{section:JGraph}

JGraph is an opensource graph visualisation library which is based on Java Swing technology. It gives the developer great means to create a generic diagram, which is consisted of nodes and edges. JGraph provides several prefabricated components for graph node shapes or for edges shapes that connect these nodes. So if you want to create simple diagram you can use these prefabricated components. Of course, there is the possibility to create your own shapes too so if you want to create diagram with nodes of some special shapes you can do it.

Except the diagram visualisation, JGraph provides some other functionality like zooming of the drawing area, undo/redo support, drag\&drop of nodes and edges and many other. This framework will be used for visualisation of the class diagram. More information about this framework can be found in \cite{jgraphmanual}.

\section{Environment}

Because this thesis is about the Netbeans plugin creation and Netbeans platform is based on the Java language, there wasn't any other option than use the Java language. But it is not bad at all. Java is widely used object oriented programming language and thanks to its portability, the application can run on a lot of operating systems like Windows, Linux and so on.

As a programming enviroment I use the Netbeans IDE. The Netbeans IDE is the best choice because it is created on the same platform as the plugin. Netbeans IDE provides the full support (Wizards, etc.) for the Netbeans plugin (or module and whole applications based on this platform) creation. The Netbeans platform is described in section \ref{section:netbeansAPI}.

For source code version control I use Git versioning system. Git is an open source distributed version control system with emphasis on the speed and efficiency. It was created by Linus Torvalds who created it for purposes of Linux kernel development. As a remote repository I use Github\footnote{web pages of Github are http://github.com/}. For more information about Git version system take a look e.g. in \cite{GitWeb}.

\include{5_application}
\chapter{Testing}

Software testing is very important part of whole development process. There has to be the quality verification in every project which is intended to be successful one. Developers has to test the application through all phases of application development. Testing has a lot of advantages. The main purpose of test is to find a bug during a development (by development I mean all phases like analysis, design, testing, etc.) or verify that all functions works as they should. 

Testing could be done manually or automatically. Manual testing is suitable in case that we don't want to repeat testing procedure a lot and thus the creation of automatic test would be harder than manual testing. Automatic tests are very helpful when we want to run them repeatedly. On the other hand, the developer who would test the same functionality again and again could miss out something important.

Automatic tests can be run whenever we want and can verify that all functions works properly. Automated test are very often used e.g. when there is a new functionality added into the application. In this case we want to verify that the new functionality did not violate any of yet existing ones. Another example can be a refactoring. Refactoring is the change of the program structure without the change of it's behavior. When we run the test before (test has to work fine before the refactoring) and after the refactoring, we can find out if the refactoring was successful (all functionality still work) or not. Of course, tests will help us only if they are created properly. It means that they test each functionality in a detail.

\section{JUnit}

For automatic test purposes I use the JUnit framework. The JUnit is well known framework which is used for unit tests creation in Java platform.

An Unit test is a test which tests the units\footnote{Unit is the smallest part of our application which can be tested alone. It means that it can be built and run.} of our application apart the others. In procedural programming this unit can be an function or a procedure. In Object Oriented programming (OOP), this unit is often a class. The Unit test then tests methods of this class. Despite of the fact that Unit tests are primarily intended to test units, they are very often used for testing of class ensemble as well. If you don't know the JUnit framework, you can find some useful information in \cite{JUnitWeb}.

The question is which parts of the application to test and which not to test. The best and simplest answer is to test everything. This approach would result in 100\% tested application but for what price? Tests would cover all functionality, which is great. But even the functionality which does not have to be tested, like implementation of user interface (UI) and so on. This means that the test creation would take a very long time and it is quite worthless.

In this project there are unit tests which covers only the business functionality of the Indepmod Class Notation plugin. These unit tests test single class functionality and on the other hand the functionality of class ensembles. These tests can be run at any time and can verify that all parts of the application works as they should. The reason why I don't test everything is that some parts of the application are not so dangerous and to test them would take a long time. 

\section{User Interface}

Among the parts that are not so dangerous belongs e.g. the User Interface (UI). An error in UI should not affect any other part of the application. Moreover, such an error would be very quickly discovered and on the other hand  to test this functionality in JUnit framework would take a very long time.

For UI, more precisely GUI\footnote{GUI - Graphic User Interface}, testing there are different tools than JUnit. Some of them are suitable for desktop application GUI testing (e.g. Abbot - more info can be found in \cite{AbbotWeb}), some of them are suitable for a web application GUI testing (e.g. Selenium - more info can be found in \cite{SeleniumWeb}), etc. 

Because I don't want to test the UI repeatedly (or not very often) I decided to test the GUI manually. Single test cases are shown in the Appendix \ref{appendixUITC} and their results are shown in this section in the Table \ref{tab:UITCResults}. I chose two environments on which I will test the application GUI:

\begin{itemize}
    \item Acer Aspire 3810TZ with Windows 7 Professional (64bit)
    \item Acer Aspire 3810TZ with Ubuntu 10.10 Maverick Meerkat (32bit)
\end{itemize}

During the development there were found some errors but they were repaired consequently. Final tests did not find any other failure. The Test Cases results are shown in next Table \ref{tab:UITCResults}.

\begin{table}[!ht]
\begin{center}
\begin{tabular}{|l|c|c|}
	\hline
	{ \bf Test Case Name } & { \bf Windows } & { \bf Ubuntu } \\
	\hline \hline
	TC0      & Passed & Passed  \\
	TC1      & Passed & Passed  \\
   TC2      & Passed & Passed  \\
   TC3      & Passed & Passed  \\
   TC4      & Passed & Passed  \\
   TC5      & Passed & Passed  \\
   TC6      & Passed & Passed  \\
   TC7      & Passed & Passed  \\
   TC8      & Passed & Passed  \\
   TC9      & Passed & Passed  \\
   TC10     & Passed & Passed  \\
   TC11     & Passed & Passed  \\
   TC12     & Passed & Passed  \\
   TC13     & Passed & Passed  \\
   TC14     & Passed & Passed  \\
   TC15     & Passed & Passed  \\
   TC16     & Passed & Passed  \\
   TC17     & Passed & Passed  \\
   TC18     & Passed & Passed  \\
   TC19     & Passed & Passed  \\
   TC20     & Passed & Passed  \\
   TC21     & Passed & Passed  \\
   TC22     & Passed & Passed  \\
   TC23     & Passed & Passed  \\
   TC24     & Passed & Passed  \\
   TC25     & Passed & Passed  \\
   TC26     & Passed & Passed  \\
   TC27     & Passed & Passed  \\
   TC28     & Passed & Passed  \\
   TC29     & Passed & Passed  \\
   TC30     & Passed & Passed  \\
   TC31     & Passed & Passed  \\
	\hline
\end{tabular}
\caption{User Interface Test Cases Results}
\label{tab:UITCResults}
\end{center}
\end{table}

\chapter{Conclusion}

\section{Future works}

Of course, there are many other things that could be done on this project. I will mention some of them, but it will not be all - the imagination has no limits. The things that I will mention could be implemented in a related bachelor, master or semestral project. These new functionalities are e.g.:

\begin{itemize}
    \item Ability to keep more information about elements (classes, interfaces, ...) and relations like its description, version, complexity, OCL\footnote{OCL - Object Constraint Language} support, and so on.
    \item The list of languages (that are displayed during the diagram creation) could be loaded from an XML (or any other) file.
    \item User Interface improvements (design and control improvements).
    \item More options of persistence (e.g. a relational database).
    \item Indepmod class notation plugin could allow to create a hierarchic structure of class diagrams (class diagrams on package level).
\end{itemize}

Because this project is a part of several related projects, there is another thing that could be done. All of these projects are aimed on the Netbeans Plugin creation. These plugins provides, like this one, the creation of an UML diagram but they are not joined together. Some other project could improve this problem. It could be done e.g. by some root plugin. The root plugin would implement the common functionality for all notations and provide services for other plugins. These other plugins would implement the functionality for particular diagram.

\section{Evaluation}

In this project I dealt with the creation of the Class modeling tool, called Indepmod Class Notation Plugin. Thanks to this project I gain an experience with new technologies like Netbeans platform and JGraph framework for which I am very thankful. The task of this project was accomplished properly. Plus, there was added another functionality like language selection in new diagram creation.

Among the project's benefit I would like to mention especially the annotation modeling support. I think that this functionality is very useful and I think that some of existing applications should implement it too (I wonder why it hasn't yet been implemented).

Among things that could have been done better is e.g. the impossibility to compose class diagrams into the hierarchical structure. This is done e.g. in the Enterprise Architect. On the other hand, this project is only at the beginning of its lifetime, so this, and another else, functionality can be implemented in future. In short, this application can not be compared with existing solutions for now, but one time, maybe, it will.




%*****************************************************************************
% Seznam literatury je v samostatnem souboru reference.bib. Ten
% upravte dle vlastnich potreb, potom zpracujte (a do textu
% zapracujte) pomoci prikazu bibtex a nasledne pdflatex (nebo
% latex). Druhy z nich alespon 2x, aby se poresily odkazy.

% originally following specification for bibliography formating was used
%\bibliographystyle{abbrv}

% Here is an improvment by Petr Dlouhy (April 2010).
% It is mainly for supervisors who expect Czech fomrating rules for references
% Additional feature is live url addresses to sources from your pdf file
% It requires the file csplainnat.bst (included in this sample zipfile).

\bibliographystyle{csplainnat}

%\bibliographystyle{plain}
%\bibliographystyle{psc}
{
%JZ: 11.12.2008 Kdo chce mit v techto ukazkovych odkazech take odkaz na CSTeX:
\def\CS{$\cal C\kern-0.1667em\lower.5ex\hbox{$\cal S$}\kern-0.075em $}
\bibliography{reference}
}

% M. Dušek radi:
%\bibliographystyle{alpha}
% kdy citace ma tvar [AutorRok] (napriklad [Cook97]). Sice to asi neni  podle ceske normy (BTW BibTeX stejne neodpovida ceske norme), ale je to nejprehlednejsi.
% 3.5.2009 JZ polemizuje: BibTeX neobvinujte, napiste a poskytnete nam styl (.bst) splnujici citacni normu CSN/ISO.

%*****************************************************************************
%*****************************************************************************
\appendix

\chapter{Package Structure of Editor Module}

You can see the package structure in this directory tree:
\renewcommand*\DTstylecomment{\rmfamily\color{black}\textsc}
\renewcommand*\DTstyle{\ttfamily\textcolor{blue}}
\dirtree{%
.1 cz.cvut.indepmod.classmodel.
.1 actions\DTcomment{Actions for buttons etc.}.
.2 nbfolders\DTcomment{Actions which are registered in layer.xml}.
.1 diagramdata\DTcomment{Classes which stores the information about the diagram}.
.2 langs\DTcomment{Definition of language specific features (e.g. the data type names)}.
.1 file\DTcomment{SaveCookie and file type association support}.
.2 wizard\DTcomment{Wizard for creation of new class model file}.
.1 frames.
.2 dialogs\DTcomment{GUI dialogs}.
.3 factory\DTcomment{Factories for some dialogs}.
.3 validation\DTcomment{Validation of values from dialogs}.
.1 persistence\DTcomment{Layer for data saving}.
.2 xml\DTcomment{implementation of persistence for saving into a XML file}.
.3 delegate\DTcomment{XML Delegates for mapping into xml}.
.1 resources\DTcomment{Bundle util}.
.1 util\DTcomment{Utility classes}.
.1 workspace\DTcomment{Main package containing the TopComponent with JGraph}.
.2 cell\DTcomment{Cells, renderers, VertexViews, etc. for JGraph}.
.3 components\DTcomment{Component for JGraph}.
.3 model.
.4 classmodel\DTcomment{Implementation of ClassModel API from API Module}.
}
\parindent=\luckyparindent

\chapter{Abstract Factory Code Example}
\label{abstractFactoryCodeExample}

\lstset{language=Java, numbers=left, rulesepcolor=\color{gray}, breaklines=true}
\begin{lstlisting}
public abstract class AbstractGUIFactory {

  private static AbstactGUIFactory linuxFactory = null;
  private static AbstractGUIFactory windowsFactory = null;

  public static AbstractGUIFactory getFactory() {
    String osName = System.getProperty("os.name");
    if (osName.startsWith("Windows")) {
      if (windowsFactory == null) {
        windowsFactory = new WindowsGUIFactory();
      }
      return windowsFactory;
    } else if (osName.equals("Linux")) {
      if (linuxFactory == null) {
        linuxFactory = new LinuxGUIFactory();
      }
      return linuxFactory;
    } else {
      //For example throw an exception
    }
  }

  public abstract JButton createButton();
  public abstract JLabel createLabel();
}

public class WindowsGUIFactory extends AbstractGUIFactory {

  public JButton createButton() {
    return new WindowsButton();
  }

  public JLabel createLabel() {
    return new WindowsLabel();
  }
}

public class LinuxGUIGactory extends AbstractGUIFactory {
  
  public JButton createButton() {
    return new LinuxButton();
  }

  public JLabel createLabel() {
    return new LinuxLabel();
  }
}
\end{lstlisting}

\chapter{Graphic User Interface Test Cases}
\label{appendixUITC}

\section{TC0 - Create class diagram}

Prerequisite: Open an Java project with an package.

\begin{itemize}
    \item Right click on a package \ding{213} new \ding{213} Other \ding{213} IndepMod \ding{213} New Class Diagram \ding{213} Next
    \item Fill Name: ClassModel, select class model in model type, Java in Language selection and click on Finish
    \item System will show an empty class diagram
    \item Open the Tool Chooser (Windows \ding{213} Tool Chooser), select a class in it and add it into the class diagram.
    \item System will show an class with name: "Class1"
    \item Right click the class and select Edit
    \item System will show an edit dialog with fields:
    \begin{itemize}
        \item Name
        \item Stereotype
        \item Abstract checkbox
        \item Annotation list with buttons for creation/edition/deletion
        \item Attribute list with buttons for creation/edition/deletion
        \item Method list with buttons for creation/edition/deletion
        \item Buttons save/cancel
    \end{itemize}
\end{itemize}

\section{TC1 - Create business diagram}

Prerequisite: Open an Java project with an package.

\begin{itemize}
    \item Right click on a package \ding{213} new \ding{213} Other \ding{213} IndepMod \ding{213} New Class Diagram \ding{213} Next
    \item Fill Name: BusinessModel, select business model in model type, Java in Language selection and click on Finish
    \item System will show an empty class diagram
    \item Open the Tool Chooser (Windows \ding{213} Tool Chooser), select a class in it and add it into the class diagram.
    \item System will show an class with name: "Class1"
    \item Right click the class and select Edit
    \item System will show an edit dialog with fields:
    \begin{itemize}
        \item Name
        \item Stereotype
        \item Attribute list with buttons for creation/edition/deletion
        \item Buttons save/cancel
    \end{itemize}
\end{itemize}

\section{TC2 - Test Elements}

Prerequisite: Created and opened an empty class diagram.

\begin{itemize}
    \item Add a class in the diagram
    \item Open an edit dialog of the class and fill the name field as 'Parent'
    \item Click on add annotation button and create an annotation with name 'version' and one attribute with name 'value' and value 'v1.1'
    \item Click on add attribute button and create an attribute with name 'firstname', data type 'String', private visibility and one annotation with name 'Length' and attribute with name 'value' and value '20'
    \item Click on add attribute button and create an attribute with name 'surname', data type 'String', private visibility and one annotation with name 'Length' and attribute with name 'value' and value '25'
    \item Click on add method button and create a method with name 'getSurname', data type 'String', public visibility and no attribute
    \item Click on add method button and create a method with name 'setSurname', data type 'void', public visibility and one attribute 'surname' of 'String' type
    \item Save the class and verify that all information are shown
    \item Create an interface and set its name as 'AnInterface', save it and verify that the name is changed
    \item Create a realisation relation between the 'Parent' class and 'AnInterface' interface; verify that the relation is shown properly
    \item Create class with name 'Child'
    \item Create a generalization between the 'Child' and 'Parent'; verify that the relation is shown properly
    \item Create an enumeration with name 'Gender' with attributes 'MALE' and 'FEMALE'; verify that enumeration is shown properly
    \item Create an association between 'Child' and 'Gender'; Set its name to 'hasGender', source cardinality to '*', target cardinality '1', unidirectional relation and visible arrows; verify that relation is shown properly
\end{itemize}

\end{document}
